% ===================================
% resume.sty
% Author: hexWars
% Version: 1.0.0
% Date: 2023-03-08
% Description: 个人简历
%  /\_/\
% ( o o )
%  > ^ <
% ===================================


\documentclass[a4paper]{article}


% 数学公式
\usepackage{amsmath}
% 插图
\usepackage{graphicx}
% 颜色
\usepackage{xcolor}
% 表格
\usepackage{array}
% 字体
\usepackage{fontspec}
% 中文
\usepackage[UTF8]{ctex}
% fontawesome图标
\usepackage{fontawesome5}
% 修改\section
\usepackage{titlesec}
% 字体大小和行间距
\usepackage{etoolbox}
% 字体格式修改
\usepackage{setspace}
% 画图
% \usepackage{tikz}
% % 修改页面样式
% \usepackage{fancyhdr}
% \pagestyle{fancy}
% \fancyhf{}
% \fancyfoot{}
% \renewcommand{\headrulewidth}{0pt}
% \renewcommand{\footrulewidth}{0pt}
% 页面布局
\usepackage[a4paper]{geometry}
% 超链接颜色修改
\usepackage[unicode, draft=false]{hyperref}

% 引入自定义包
\usepackage{resume}


% ===================================
% 简历正文
% ===================================

\begin{document}

% 去掉页码
\pagestyle{empty}
% 去除缩进,article的正文每段会默认缩进
\setlength{\parindent}{0pt}

% 修改字体大小和行间距
{\fontsize{9pt}{7pt}\selectfont

% ===================================
% 社交信息
% ===================================

\begin{center}
    \textbf{\Huge 克莱恩} \\ 
    \vspace{5pt} % 行间距
    \hspace{10pt}
    \href{tel:+8600000000000}{\faPhoneVolume \ 00000000000} | 
    \href{weixin:666}{\faWeixin \ 666} | 
    \href{https://github.com}{\faGithub \ github}  | 
    \href{mailto:email}{\faEnvelope \ e-mail} | 
    \href{https://blog.xxx.xxx/}{\faRss \ blog.xxx.xxx}
    \vspace{-5pt}
\end{center}

%-------------------------------------------------------------------------------
%	教育经历
%-------------------------------------------------------------------------------

\ignorespaces
\section{\textbf{教育经历}}
    \resumeTable
    {霍格沃兹}{2019年9月 -- 2023年6月}
    {软件工程 \ 本科 \ 信息技术学院}{GPA:3.5(前\%5)}

%-------------------------------------------------------------------------------
%	专业技能
%-------------------------------------------------------------------------------

\section{\textbf{专业技能}}

    \textbullet 精通常见数据结构及算法,具有良好的编程习惯 \\
    \textbullet 精通SpringBoot和Node.js进行后台开发 \\
    \textbullet 精通Linux等操作系统的使用 \\
    \textbullet 精通HTML,CSS,JavaScript的使用 \\
    \textbullet 精通JSP,Thymeleaf,FreeMarker等模板引擎 \\
    \textbullet 精通DevOps \\
    \textbullet 精通各种消息队列 \\
    \textbullet 精通Mysql和Redis等数据库 \\
    \textbullet 精通Java,Go等后端语言 \\
    \textbullet 精通Vue,React等前端框架 \\

    大一开始在CSDN发表技术文章,至今约300篇,浏览量约13w,全站排名最高约\textbf{5000}, \href{https://blog.csdn.net}{博客地址}

%-------------------------------------------------------------------------------
%	荣誉奖项
%-------------------------------------------------------------------------------

\section{\textbf{荣誉奖项}}

    \textbullet 大学生创新创业项目(国家级)两项 \\
    \textbullet 专业一等奖学金(三次),校优秀学生干部(一次) \\
    \textbullet 某大学XX校区“互联网++”大学生创新创业大赛“红旅创意组”金奖 \hfill 2022年6月 \\
    \textbullet 国家软件著作权登记证书(一项)\hfill 2022年5月 \\
    \textbullet 十三届蓝桥杯省赛一等奖 \hfill 2022年4月 \\
    \textbullet 某比赛满分(1/982) \hfill 2021年12月 \\
    \textbullet 第九届“逐鹿今朝”大学生创业计划竞赛(校赛)三等奖 \hfill 2021年12月 \\
    \textbullet (第三届)算法设计与编程挑战赛(秋季赛)个人赛组银奖 \hfill 2021年11月

\ignorespaces

%-------------------------------------------------------------------------------
%	社团和组织经历
%-------------------------------------------------------------------------------

\section{\textbf{社团和组织经历}}
    
    \resumeTable
    {程序设计爱好者协会}{2020年10月 -- 2021年10月}
    {会长}{}
    负责校内程序设计竞赛的出题和测题以及协会招新,协助学院举办XXX大赛等多项大型活动

    \vspace{2pt} 同时管理宣传部,技术部,竞赛部,组织部近40人,任期结束时在学生社团评比中所在社团获得总评\textbf{第一}的成绩 


%-------------------------------------------------------------------------------
%	实习经验
%-------------------------------------------------------------------------------

\section{\textbf{实习经验}}
    \resumeTable
    {XXX研发中心}{2000年2月 -- 2020年2月}
    {后端工程师}{北京}
    
    \begin{minipage}{17cm}\vspace{2pt}
        \textbf{工作描述:}本项目针对XXX的场景进行开发工作,其中包括发布删除修改查找,指定信息展示,分类目录,登录模块权限管理,登录日志等功能。从数据库表设计开始,文章管理和目录分类接口的实现 ,最后打包成Docker镜像方便以Docker方式部署
    \end{minipage}

    \vspace{7pt}% 修改行间距

    \resumeTable
    {XXX公司}{2023年2月 -- 至今}
    {后端工程师}{广州}
    
    \begin{minipage}{17cm}\vspace{2pt}
        \textbf{工作描述:}本项目针对XXX的场景进行开发工作,其中包括发布删除修改查找,指定信息展示,分类目录,登录模块权限管理,登录日志等功能。从数据库表设计开始,文章管理和目录分类接口的实现 ,最后打包成Docker镜像方便以Docker方式部署
    \end{minipage}

%-------------------------------------------------------------------------------
%	项目经历
%-------------------------------------------------------------------------------

\section{\textbf{项目经历}}
    \resumeTable
    {XXX系统}{2021年10月 -- 2021年11月}
    {全栈开发}{}
    \begin{minipage}{17cm}\vspace{2pt}
        \textbf{技术选型:}Node.js+Express+EJS+Mysql+Docker
    \end{minipage}
    

    \begin{minipage}{17cm}\vspace{2pt}
        \textbf{项目描述:}本项目针对XXX的场景进行开发工作,其中包括发布删除修改查找,指定信息展示,分类目录,登录模块权限管理,登录日志等功能。从数据库表设计开始,文章管理和目录分类接口的实现 ,最后打包成Docker镜像方便以Docker方式部署
    \end{minipage}

    \begin{minipage}{17cm}\vspace{2pt}
        \textbf{职责描述:}负责项目的总体设计,参考XXX进行原型设计,完成XXX代码,并制作Docker镜像上传
    \end{minipage}

    \vspace{7pt}% 修改行间距

    \resumeTable
    {XXX系统}{2021年10月 -- 2021年11月}
    {全栈开发}{}
    \begin{minipage}{17cm}\vspace{2pt}
        \textbf{技术选型:}Node.js+Express+EJS+Mysql+Docker
    \end{minipage}
    

    \begin{minipage}{17cm}\vspace{2pt}
        \textbf{项目描述:}本项目针对XXX的场景进行开发工作,其中包括发布删除修改查找,指定信息展示,分类目录,登录模块权限管理,登录日志等功能。从数据库表设计开始,文章管理和目录分类接口的实现 ,最后打包成Docker镜像方便以Docker方式部署
    \end{minipage}

    \begin{minipage}{17cm}\vspace{2pt}
        \textbf{职责描述:}负责项目的总体设计,参考XXX进行原型设计,完成XXX代码,并制作Docker镜像上传
    \end{minipage}



%-------------------------------------------------------------------------------
%	自我评价
%-------------------------------------------------------------------------------


\section{\textbf{自我评价}}
执着,喜欢思考,爱好广泛,拥有冷静的内心和炽热的灵魂,独特的思维角度 \\
从第一次AC开始,从第一次发博客开始,在思考中不断学习,希望能和更加优秀的人一起共事 \\
热爱开源,热爱技术,对代码的美感深深的吸引

\vfill
\center{\footnotesize Last updated: \today}

\end{document}
